

% Modelo de artigo para ser utilizado na aula de Latex - PCALL
% Deverá ser editado apenas a partir da linha 100

%informações do documento
\documentclass[
	article,			% indica que é um artigo acadêmico
	11pt,				% tamanho da fonte
	oneside,			% para impressão apenas no verso. Oposto a twoside
	a4paper,			% tamanho do papel.
	% -- opções do pacote babel --
	english,			% idioma adicional para hifenização
	brazil,				% o último idioma é o principal do documento
	sumario=tradicional
	]{abntex2}

% ---
% Pacotes fundamentais
% ---
\usepackage{lmodern}			% Usa a fonte Latin Modern
\usepackage[T1]{fontenc}		% Selecao de codigos de fonte.
\usepackage[utf8]{inputenc}		% Codificacao do documento (conversão automática dos acentos)
\usepackage{indentfirst}		% Indenta o primeiro parágrafo de cada seção.
\usepackage{nomencl} 			% Lista de simbolos
\usepackage{color}				% Controle das cores
\usepackage{graphicx}			% Inclusão de gráficos
\usepackage{microtype} 			% para melhorias de justificação
% ---

% Pacotes adicionais, usados apenas no âmbito do Modelo Canônico do abnteX2
% ---
\usepackage{lipsum}				% para geração de dummy text

% Pacotes de citações
% ---
\usepackage[brazilian,hyperpageref]{backref}	 % Paginas com as citações na bibl
\usepackage[alf]{abntex2cite}	% Citações padrão ABNT
% ---
% Configurações do pacote backref
% Usado sem a opção hyperpageref de backref
\renewcommand{\backrefpagesname}{Citado na(s) página(s):~}
% Texto padrão antes do número das páginas
\renewcommand{\backref}{}
% Define os textos da citação
\renewcommand*{\backrefalt}[4]{
	\ifcase #1 %
		Nenhuma citação no texto.%
	\or
		Citado na página #2.%
	\else
		Citado #1 vezes nas páginas #2.%
	\fi}%
% ---
% Configurações de aparência do PDF final
% alterando o aspecto da cor azul
\definecolor{blue}{RGB}{41,5,195}

% informações do PDF
\makeatletter
\hypersetup{
     	%pagebackref=true,
		pdftitle={\@title},
		pdfauthor={\@author},
    	pdfsubject={Modelo de artigo científico com abnTeX2},
	    pdfcreator={LaTeX with abnTeX2},
		pdfkeywords={abnt}{latex}{abntex}{abntex2}{atigo científico},
		colorlinks=true,       		% false: boxed links; true: colored links
    	linkcolor=blue,          	% color of internal links
    	citecolor=blue,        		% color of links to bibliography
    	filecolor=magenta,      		% color of file links
		urlcolor=blue,
		bookmarksdepth=4
}
\makeatother
% ---
% compila o indice
% ---
\makeindex
% ---
% Altera as margens padrões
% ---
\setlrmarginsandblock{3cm}{3cm}{*}
\setulmarginsandblock{3cm}{3cm}{*}
\checkandfixthelayout
% ---
% ---
% Espaçamentos entre linhas e parágrafos
% ---
% O tamanho do parágrafo é dado por:
\setlength{\parindent}{1.3cm}

% Controle do espaçamento entre um parágrafo e outro:
\setlength{\parskip}{0.2cm}  % tente também \onelineskip

% Espaçamento simples
\SingleSpacing

% ---
% ---
% Informações de dados para CAPA e FOLHA DE ROSTO
% ---
\titulo{Trabalho de Programacao Concorrente \\
Lista Encadeada}
\autor{Eduardo Souza Paixão}
\local{Brasil}
\data{}
% ---

\ifthenelse{\equal{\ABNTEXisarticle}{true}}{%
\renewcommand{\maketitlehookb}{}
}{}

% ----
% Início do documento
% ----
\begin{document}

% Retira espaço extra obsoleto entre as frases.
\frenchspacing

\maketitle


\section*{Introdução}

A fim de praticar os conceitos e mecanismos de sincronização expostos
durante o recorrer da disciplina, será apresentada a implementação da
estrutura de dados "lista encadeada" de forma concorrente, utilizando a
linguagem \textit{C++}.

\section*{Descrição do problema}

A lista encadeada consiste em uma sequencia de nós encadeados formando
uma lista, estes nós são compostos por um dado e um ponteiro contendo a
referência de um proximo nó. As operações realizadas por essa lista são
as seguintes:

Dentre os mecanismos de sincronização apresentados em aula, foi escolhido o
metodo de exclusão mutua para efetuar a sincronização entre as threads. No
\textit{C++} essa mecanismo pode ser utilizado por meio da biblioteca
mutex. Entretanto, foi utilizado uma abstracao desse conceito por meio
do "lock\_guard" que habilita a exclusão.

\begin{itemize}
  \item{\textbf{insert}}: Insere um nó no final da lista.

  \item{\textbf{remove}}: Remove um nó se presente em uma posição determinada.

  \item{\textbf{search}}: Busca um nó na lista.
\end{itemize}

De acordo com as especificações do projeto, foram estipulados tipos de
threads para cada uma das operações supracitadas.

\subsection{insert}
% std::lock_guard<std::mutex> lock(m_mutex);
\subsection{remove}
% std::lock_guard<std::mutex> lock(m_mutex);

\subsection{search}
% std::lock_guard<std::mutex> lock(m_mutex);



\section*{Conclusao}
\addcontentsline{toc}{section}{Uso de threads}
A implementacao da estrutura proposta de forma concorrente me fez
perceber que as implicacoes da programacao concorrente, como race
conditions em lugares inesperados falsos positivos em relacao a
sincronização de threads dentre outras questoes que me farao estar
atento ao programar de forma concorrente.

%Final do Documento
\end{document}
