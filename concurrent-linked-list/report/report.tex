

% Modelo de artigo para ser utilizado na aula de Latex - PCALL
% Deverá ser editado apenas a partir da linha 100

%informações do documento
\documentclass[
	article,			% indica que é um artigo acadêmico
	11pt,				% tamanho da fonte
	oneside,			% para impressão apenas no verso. Oposto a twoside
	a4paper,			% tamanho do papel.
	% -- opções do pacote babel --
	english,			% idioma adicional para hifenização
	brazil,				% o último idioma é o principal do documento
	sumario=tradicional
	]{abntex2}

% ---
% Pacotes fundamentais
% ---
\usepackage{lmodern}			% Usa a fonte Latin Modern
\usepackage[T1]{fontenc}		% Selecao de codigos de fonte.
\usepackage[utf8]{inputenc}		% Codificacao do documento (conversão automática dos acentos)
\usepackage{indentfirst}		% Indenta o primeiro parágrafo de cada seção.
\usepackage{nomencl} 			% Lista de simbolos
\usepackage{color}				% Controle das cores
\usepackage{graphicx}			% Inclusão de gráficos
\usepackage{microtype} 			% para melhorias de justificação
% ---

% Pacotes adicionais, usados apenas no âmbito do Modelo Canônico do abnteX2
% ---
\usepackage{lipsum}				% para geração de dummy text

% Pacotes de citações
% ---
\usepackage[brazilian,hyperpageref]{backref}	 % Paginas com as citações na bibl
\usepackage[alf]{abntex2cite}	% Citações padrão ABNT
% ---
% Configurações do pacote backref
% Usado sem a opção hyperpageref de backref
\renewcommand{\backrefpagesname}{Citado na(s) página(s):~}
% Texto padrão antes do número das páginas
\renewcommand{\backref}{}
% Define os textos da citação
\renewcommand*{\backrefalt}[4]{
	\ifcase #1 %
		Nenhuma citação no texto.%
	\or
		Citado na página #2.%
	\else
		Citado #1 vezes nas páginas #2.%
	\fi}%
% ---
% Configurações de aparência do PDF final
% alterando o aspecto da cor azul
\definecolor{blue}{RGB}{41,5,195}

% informações do PDF
\makeatletter
\hypersetup{
     	%pagebackref=true,
		pdftitle={\@title},
		pdfauthor={\@author},
    	pdfsubject={Modelo de artigo científico com abnTeX2},
	    pdfcreator={LaTeX with abnTeX2},
		pdfkeywords={abnt}{latex}{abntex}{abntex2}{atigo científico},
		colorlinks=true,       		% false: boxed links; true: colored links
    	linkcolor=blue,          	% color of internal links
    	citecolor=blue,        		% color of links to bibliography
    	filecolor=magenta,      		% color of file links
		urlcolor=blue,
		bookmarksdepth=4
}
\makeatother
% ---
% compila o indice
% ---
\makeindex
% ---
% Altera as margens padrões
% ---
\setlrmarginsandblock{3cm}{3cm}{*}
\setulmarginsandblock{3cm}{3cm}{*}
\checkandfixthelayout
% ---
% ---
% Espaçamentos entre linhas e parágrafos
% ---
% O tamanho do parágrafo é dado por:
\setlength{\parindent}{1.3cm}

% Controle do espaçamento entre um parágrafo e outro:
\setlength{\parskip}{0.2cm}  % tente também \onelineskip

% Espaçamento simples
\SingleSpacing

% ---
% ---
% Informações de dados para CAPA e FOLHA DE ROSTO
% ---
\titulo{Trabalho de Programação Concorrente}
\autor{Eduardo Souza Paixão}
\local{Brasil}
\data{}
% ---

\ifthenelse{\equal{\ABNTEXisarticle}{true}}{%
\renewcommand{\maketitlehookb}{}
}{}

% ----
% Início do documento
% ----
\begin{document}

% Retira espaço extra obsoleto entre as frases.
\frenchspacing

\maketitle


\section{Introdução}

A fim de praticar os conceitos expostos durante o decorrer da disciplina, será
apresentada a implementação da estrutura de dados lista encadeada de forma
concorrente, utilizando a linguagem \textit{C++}.

\section{Descrição do problema}

A estrutura a ser implementada consiste em uma sequência de nós
encadeados formando uma lista, estes nós são compostos por um dado, e
um ponteiro contendo a referência para um nó seguinte.

De acordo com as especificações do projeto, a lista deverá executar as
operações de inserção, remoção e busca, utilizando grupos de threads
específicos para cada uma das operações. Bem como mecanismos
sincronização, visando eliminar os problemas de starvation e deadlock em
consequência do compartilhamento de recurso pelas threads.

\section{Implementação}
Para a implementação, foi criada uma classe ``Linked\_list'' reponsável por
implementar uma lista encadeada simples. Em sequida foi desenvolvida a classe
``Concurrent\_linked\_list'' que herda Linked\_list e encapsula seus
métodos aplicando as threads e os mecanismos de sincornização.

A sincronizção das threads se deu pelo uso do método de exclusão mutua,
que foi utilizado por meio da classe ``lock\_guard'' que atua
protegendo uma região crítica até o final do seu escopo de execução.
Assim, evitando problemas como deadlock e starvation.

\subsection{Inserção}

Esta operação se dá por meio da adição de um nó no final da lista.
Para evitar que duas threads possam realizar esta operação ao mesmo tempo
foi feito o uso do lock\_guard, restringindo o acesso a esse metodo a
uma threads por vez.

\begin{figure}[!h]
\includegraphics[width=10cm]{img/concurrency-insertion-pic.png}
\centering
\end{figure}

\subsection{Remoção}

Ao passar um número aleatório para esta função, se este número
representar a posição de um elemento na lista, este elemento será
removido. Semelhante ao caso da inserção, neste método também foi
aplicado o lock\_guard.

\begin{figure}[!h]
\includegraphics[width=10cm]{img/concurrency-remotion-pic.png}
\centering
\end{figure}

\subsection{Busca}
Essa operação consiste na busca por um determinado elemento no campo
de dados dos nós da lista encadeada. Diferentemente dos dos métodos
anteriores, a busca não altera o estado da lista, fazendo com que não
seja necessário aplicar nenhuma restrição nas threads que executrão
esta função.

\begin{figure}[!h]
\includegraphics[width=10cm]{./img/concurrentcy-search-pic.png}
\centering
\end{figure}


\section{Considerações}
Eu não entendi bem, como funcionaría a política de justiça descrita
entre as threads dos grupos inserção e remoção na descrição do projeto,
mas pelo que eu pude apurar, ambos os grupos de threads estão cumprindo
suas funções sem interferir umas nas outras.

No mais, foi bastante interessante ver como os mecanismos de
sincronização funcionam e como a disputadas das threads por recursos
compartilhados pode gerar incosistencias graves no funcionamento de um
software.


\end{document}
